\chapter*{Preface}\normalsize
\addcontentsline{toc}{chapter}{Preface}
\pagestyle{plain}

\textbf{OpenIMAJ} is a set of libraries and tools for multimedia analysis. 
OpenIMAJ is very broad and contains everything from state-of-the-art computer 
vision (e.g. SIFT descriptors, salient region detection, face detection, etc.) 
and advanced data clustering, through to software that performs analysis on the 
layout and structure of webpages.

OpenIMAJ is primarily written in pure Java and, as such, is completely platform 
independent. The video capture and hardware libraries contain some native code 
but Linux (x86 and x86\_64 are supported currently; ARM support is coming soon), 
OSX and Windows are supported out of the box (under both 32 and 64 bit JVMs). 
It is possible to write programs that use the libraries in any JVM language 
that supports Java interoperability, such as Groovy, Jython, JRuby or 
Scala. OpenIMAJ can even be run on Android phones and tablets.

The OpenIMAJ software is structured into a number of modules. The modules 
can be used independently, so if, for instance, you were developing data 
clustering software using OpenIMAJ you wouldn't need to acquire the modules related 
to images. The list on the following page illustrates the modules 
and summarises the functionality in each component.

This tutorial aims to instruct the reader on how to get up and running
writing code using OpenIMAJ. Currently the tutorial covers the following areas:
\begin{enumerate}
	\item Getting started with OpenIMAJ using Maven
	\item Processing your first image
	\item Introduction to clustering, segmentation and connected components
	\item Processing video
	\item Finding faces
	\item Global image features
	\item SIFT and feature matching
\end{enumerate}

In the future we hope to add more content to the tutorial covering the following:
\begin{itemize}
	\item Image and video indexing using ImageTerrier
	\item Compiling OpenIMAJ from source
	\item Tracking features in video
	\item Audio processing
	\item Hardware interfaces
	\item Advanced local features
	\item Scalable processing with OpenIMAJ/Hadoop
	\item Machine learning
\end{itemize}

\section*{The OpenIMAJ Modules}

\begin{figure*}[h!]
\renewcommand*\DTstylecomment{\rmfamily{ }{ }}
\newcommand{\descwidth}{8.0cm} 
\DTsetlength{0.2em}{0.4em}{0.2em}{0.4pt}{1.6pt}
\dirtree{%
.1 openimaj.
.2 core \DTcomment{\begin{minipage}[t]{\descwidth}
Submodule for modules containing functionality used across the library.
\end{minipage}}.
.3 core \DTcomment{\begin{minipage}[t]{\descwidth}
Core library functionality concerned with general programming problems rather than multimedia specific functionality. Includes I/O utilities, randomisation, hashing and type conversion.
\end{minipage}}.
.3 feature \DTcomment{\begin{minipage}[t]{\descwidth}
Core notion of features, usually denoted as arrays of data. Definitions of features for all primitive types, features with location and lists of features (both in memory and on disk).
\end{minipage}}.
.3 audio \DTcomment{\begin{minipage}[t]{\descwidth}
Core definitions of audio streams and samples/chunks. Also contains interfaces for processors for these basic types.
\end{minipage}}.
.3 image \DTcomment{\begin{minipage}[t]{\descwidth}
Core definitions of images, pixels and connected components. Also contains interfaces for processors for these basic types.
\end{minipage}}.
.3 video \DTcomment{\begin{minipage}[t]{\descwidth}
Core definitions of a video type and functionality for displaying and processing videos.
\end{minipage}}.
.3 video-capture \DTcomment{\begin{minipage}[t]{\descwidth}
Cross-platform video capture interface using a lightweight native interface. Supports 32 and 64 bit JVMs under Linux, OSX and Windows.
\end{minipage}}.
.3 math \DTcomment{\begin{minipage}[t]{\descwidth}
Mathematical implementations including geometric, matrix and statistical operators. 
\end{minipage}}.
.2 audio \DTcomment{\begin{minipage}[t]{\descwidth}
Submodule for audio related functionality.
\end{minipage}}.
.3 processing \DTcomment{\begin{minipage}[t]{\descwidth}
Implementations of various audio processors (e.g. multichannel conversion, volume change, ...).
\end{minipage}}.
.2 ....
}
\end{figure*}

\begin{figure*}[h!]
\renewcommand*\DTstylecomment{\rmfamily{ }{ }}
\newcommand{\descwidth}{8.0cm} 
\DTsetlength{0.2em}{0.4em}{0.2em}{0.4pt}{1.6pt}
\dirtree{%
.1 ....
.2 image \DTcomment{\begin{minipage}[t]{\descwidth}
Submodule for image related functionality.
\end{minipage}}.
.3 processing \DTcomment{\begin{minipage}[t]{\descwidth}
Implementations of various image, pixel and connected component processors (resizing, convolution, edge detection, ...).
\end{minipage}}.
.3 feature-extraction \DTcomment{\begin{minipage}[t]{\descwidth}
Methods for the extraction of low-level image features, including global image features and pixel/patch classification models.
\end{minipage}}.
.3 local-features \DTcomment{\begin{minipage}[t]{\descwidth}
Methods for the extraction of local features. Local features are descriptions of regions of images (SIFT, ...) selected by detectors (Difference of Gaussian, Harris, ...). 
\end{minipage}}.
.3 faces \DTcomment{\begin{minipage}[t]{\descwidth}
Implementation of a flexible face-recognition pipeline, including pluggable detectors, aligners, feature extractors and recognisers.
\end{minipage}}.
.2 machine-learning \DTcomment{\begin{minipage}[t]{\descwidth}
Algorithms which aid the classification and search of data.
\end{minipage}}.
.3 nearest-neighbour \DTcomment{\begin{minipage}[t]{\descwidth}
Brute force and KD-Tree implementations of exact and approximate KNN.
\end{minipage}}.
.3 clustering \DTcomment{\begin{minipage}[t]{\descwidth}
Various clustering algorithm implementations for all primitive types including random, random forest, \mbox{K-Means} (Exact, Hierarchical and Approximate), ...
\end{minipage}}.
.2 hadoop \DTcomment{\begin{minipage}[t]{\descwidth}
Extensions to enable interaction with the Apache Hadoop Map-Reduce implementation.
\end{minipage}}.
.3 core-hadoop \DTcomment{\begin{minipage}[t]{\descwidth}
Reusable wrappers to access and create sequence-files and map-reduce jobs.
\end{minipage}}.
.3 tools \DTcomment{\begin{minipage}[t]{\descwidth}
Tools that provide Map-Reduce jobs that can be run on a Hadoop cluster.
\end{minipage}}.
.4 HadoopClusterQuantiserTool \DTcomment{\begin{minipage}[t]{\descwidth}
Distributed feature quantisation tool.
\end{minipage}}.
.4 HadoopFastKMeans \DTcomment{\begin{minipage}[t]{\descwidth}
Distributed feature clustering tool.
\end{minipage}}.
.4 HadoopGlobalFeaturesTool \DTcomment{\begin{minipage}[t]{\descwidth}
Distributed global image feature extraction tool.
\end{minipage}}.
.4 HadoopImageDownload \DTcomment{\begin{minipage}[t]{\descwidth}
Distributed image download tool.
\end{minipage}}.
.4 HadoopLocalFeaturesTool \DTcomment{\begin{minipage}[t]{\descwidth}
Distributed local image feature extraction tool.
\end{minipage}}.
.4 SequenceFileIndexer \DTcomment{\begin{minipage}[t]{\descwidth}
Tool for building an index of the keys in a Hadoop SequenceFile.
\end{minipage}}.
.4 SequenceFileTool \DTcomment{\begin{minipage}[t]{\descwidth}
Tool for building, inspecting and extracting Hadoop SequenceFiles.
\end{minipage}}.
.2 hardware \DTcomment{\begin{minipage}[t]{\descwidth}
Various interfaces to hardware devices that we've used in projects built using OpenIMAJ.
\end{minipage}}.
.3 serial \DTcomment{\begin{minipage}[t]{\descwidth}
Interface to hardware devices that connect to serial or USB-serial ports.
\end{minipage}}.
.3 gps \DTcomment{\begin{minipage}[t]{\descwidth}
Interface to GPS devices that support the NMEA protocol.
\end{minipage}}.
.3 compass \DTcomment{\begin{minipage}[t]{\descwidth}
Interface to an OceanServer OS5000 digital compass.
\end{minipage}}.
.2 ....
}
\end{figure*}

\begin{figure*}[h!]
\renewcommand*\DTstylecomment{\rmfamily{ }{ }}
\newcommand{\descwidth}{8.0cm} 
\DTsetlength{0.2em}{0.4em}{0.2em}{0.4pt}{1.6pt}
\dirtree{%
.1 ....
.2 web \DTcomment{\begin{minipage}[t]{\descwidth}
Support for analysing and processing web-pages.
\end{minipage}}.
.3 core-web \DTcomment{\begin{minipage}[t]{\descwidth}
Implementation of a programatic offscreen web browser and utility functions.
\end{minipage}}.
.3 analysis \DTcomment{\begin{minipage}[t]{\descwidth}
Utilities for analysing the content and visual layout of a web-page.
\end{minipage}}.
.2 video \DTcomment{\begin{minipage}[t]{\descwidth}
Support for analysing and processing video.
\end{minipage}}.
.3 video-processing \DTcomment{\begin{minipage}[t]{\descwidth}
Various video processing algorithms, such as shot-boundary detection.
\end{minipage}}.
.3 xuggle-video \DTcomment{\begin{minipage}[t]{\descwidth}
Plugin to use Xuggler as a video source. Allows most video formats to be read into OpenIMAJ.
\end{minipage}}.
.2 thirdparty \DTcomment{\begin{minipage}[t]{\descwidth}
Thirdparty code that has been integrated into OpenIMAJ.
\end{minipage}}.
.3 klt-tracker \DTcomment{\begin{minipage}[t]{\descwidth}
Implementation of the Kanade-Lucus-Tomasi feature tracker.
\end{minipage}}.
.2 tools \DTcomment{\begin{minipage}[t]{\descwidth}
Commandline tools exposing OpenIMAJ functionality
\end{minipage}}.
.3 CityLandscapeClassifier \DTcomment{\begin{minipage}[t]{\descwidth}
Tool for classifying images as being cityscapes/landscapes (or natural/unnatural).
\end{minipage}}.
.3 FaceTools \DTcomment{\begin{minipage}[t]{\descwidth}
Tools for face detection and recognition.
\end{minipage}}.
.3 FeatureVisualisation \DTcomment{\begin{minipage}[t]{\descwidth}
Tools for visualising certain types of image feature.
\end{minipage}}.
.3 FlickrCrawler \DTcomment{\begin{minipage}[t]{\descwidth}
Tool for downloading image datasets from Flickr.
\end{minipage}}.
.3 GlobalFeaturesTool \DTcomment{\begin{minipage}[t]{\descwidth}
Tool for extracting global features from images.
\end{minipage}}.
.3 ImageCollectionTool \DTcomment{\begin{minipage}[t]{\descwidth}
Tool for creating collections of images from various sources.
\end{minipage}}.
.3 LocalFeaturesTool \DTcomment{\begin{minipage}[t]{\descwidth}
Tool for extracting local features from images.
\end{minipage}}.
.3 OCRTools \DTcomment{\begin{minipage}[t]{\descwidth}
Tool for applying OCR to images.
\end{minipage}}.
.3 WebTools \DTcomment{\begin{minipage}[t]{\descwidth}
Tools for extracting and analysing the layout and visual characteristics of webpages.
\end{minipage}}.
}
\end{figure*}
